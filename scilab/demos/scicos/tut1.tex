Tutorial: constructing a simple diagram.
Scicos editor is launched by the Scilab command:

scicos

This opens up the editor window with an empty diagram
named Untitled
To construct a diagram, we start by opening the
Sources palette .
To copy a block from the palette, we click on the
block (in the palette) and then click in the diagram
where the block is to be placed.
The square wave generator is copied into
the diagram. This block generates -1 and +1
alternatively. The output changes everytime
the block is activated. To activate the block,
we need a source of activation. We use an event
clock.
To connect the two blocks, we click close to the
output port and then close to the destination port.

To change block parameters, simply click on the block.
Let us now adjust the parameters of the clock.
The integrator block is copied from the Linear palette.
We now connect the output of the square wave generator to
to the input of the integrator.
This is done by clicking near the output port of the generator,
then on intermediary points to direct the link, and finally
near the input port of the integrator.
This diagram can be simulated now. To visulaize simulation results,
we need a scope. Scopes can be found in the Sink palette.
The scope needs to be activated. At each activation,
the value at the inputs of the scope are read and
used in the plot. If we use the clock generator in
the diagram, we don't see what happens between two
events. That is why we use a faster clock.
Simulation settings can be set by using the
Setup item in the Simulation menu. The final
simulation time is set to 30.
We notice that the simulation is very slow. The reason
is that we have chosen a very small clock period for
activating the scope. We can increase the period (reduce
the number of points in the plot) but we can also bufferize
the output of the scope. We also change the y-axes parameters
of the scope to get a better picture of the simulation result.
We see that the simulation is much faster now.

Let us now replace the integrator with a more
complex linear system.
The linear system is defined in terms
of the transfer function. The default
value is 1/(s+1).
We now change the transfer function.
To save the diagram for future use, we can simply save it
using the Save item in the Diagram menu. But that saves the
diagram in the file Untitled.cos; we don't get a chance to
choose the name. For that we can use the "Save As" button,
or rename the diagram before saving it.
We can leave Scicos using the Quit button or
simply closing the window (click on the upper
right hand X).
