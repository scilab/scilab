%
% scilabstats.tex --
%   Some notes about Scilab statistical features in Scilab.
%
% Copyright 2008 Michael Baudin
%
\documentclass[12pt]{report}

%% Good fonts for PDF
\usepackage[cyr]{aeguill}

%% Package for page headers
\usepackage{fancyhdr}

%% Package to include graphics
%% Comment for DVI
\usepackage[pdftex]{graphicx}

%% Figures formats: jpeg or pdf
%% Comment for DVI
\DeclareGraphicsExtensions{.jpg,.pdf}

%% Package to create Hyperdocuments
%% Comment for DVI
\usepackage[pdftex,colorlinks=true,linkcolor=blue,citecolor=blue,urlcolor=blue]{hyperref}

%% Package to control printed area size
\usepackage{anysize}
%% ...by defining margins {left}{right}{top}{bottom}
\marginsize{22mm}{14mm}{12mm}{25mm}

%% Package used to include a bibliography
\usepackage{natbib}

%% R for real numbers
\usepackage{amssymb}

%% User defined commands

\usepackage{url}

% Scilab macros
\newcommand{\scimacro}[1]{\textit{#1}}
\newcommand{\scicommand}[1]{\textit{#1}}

% To highlight source code
\usepackage{listings}

% Define theorem environments 
\newtheorem{theorem}{Theorem}[section]
\newtheorem{lemma}[theorem]{Lemma}
\newtheorem{proposition}[theorem]{Proposition}
\newtheorem{corollary}[theorem]{Corollary}

\newenvironment{proof}[1][Proof]{\begin{trivlist}
\item[\hskip \labelsep {\bfseries #1}]}{\end{trivlist}}
\newenvironment{definition}[1][Definition]{\begin{trivlist}
\item[\hskip \labelsep {\bfseries #1}]}{\end{trivlist}}
\newenvironment{example}[1][Example]{\begin{trivlist}
\item[\hskip \labelsep {\bfseries #1}]}{\end{trivlist}}
\newenvironment{remark}[1][Remark]{\begin{trivlist}
\item[\hskip \labelsep {\bfseries #1}]}{\end{trivlist}}

\newcommand{\qed}{\nobreak \ifvmode \relax \else
      \ifdim\lastskip<1.5em \hskip-\lastskip
      \hskip1.5em plus0em minus0.5em \fi \nobreak
      \vrule height0.75em width0.5em depth0.25em\fi}

% Maths shortcuts 
\newcommand{\RR}{\mathbb{R}}

% Algorithms
\usepackage{algorithm2e}

\begin{document}
\author{Michael Baudin}
\date{February 2009}
\title{Scilab and Statistics}
\begin{abstract}
In this document, we describe the statistical features of Scilab.
We analyse the features available in Scilab's core (i.e. provided
"out of the box") and Scilab Statistical Toolboxes.
For Scilab's core statistical features, we analyse the different
libraries used by Scilab and provide a complete overview of 
the functions. For the most important features, we present Scilab 
sessions with a sample use of the command. Several Scilab Toolboxes
are analysed in this document, including Sci\_R and Stixbox.
We also analyse the missing features (not provided in the core and not in the 
toolboxes) with the tools which are available in other languages,
including Matlab and R.
\end{abstract}

\maketitle

\tableofcontents

\chapter{Introduction}

As stated in \cite{scilabstats2001}, Scilab's core provides
a complete set of features related to simulation and statistical
computations.
Indeed, Scilab provides uniform pseudo-random number generators, functions to 
compute the moments of a distribution and a complete set of 
distributions. In this document, we will make a complete 
overview on these features.

It must be noticed, though, that these features
are not as complete as in other languages, like R for example.
This is why several toolboxes have developped in order to extend 
the features of Scilab. In this document, we will present two
major toolboxes, that is the Sci\_R toolbox and the Stix toolbox.

In the last chapter, we will analyse the missing statistical 
features and will analyse how these features are available in other
tools, such as Matlab, R, or Octave.

\section{A sample session}

A good introduction on the statistical features of Scilab is \cite{scilabintro2007}.
In the remaining of this introduction chapter, we will try to have 
a flavour of how to perform statistical computations with Scilab.
We focus on the algorithms which are used inside Scilab, to show what 
exact algorithms perform the computations.

As a first example, we will generate a sequence of numbers from a 
normal law with mean 0 and standard deviation 1 (example inspired and simplified 
from \cite{scilabintro2007}). The probability density function (pdf) 
and the cumulated probability density function of the normal law is 
\begin{eqnarray}
f(x) &=& \frac{1}{\sqrt{2\pi}} e^{-\frac{t^2}{2}},\\
P(x) &=& \frac{1}{\sqrt{2\pi}} \int_{-\infty}^x e^{-\frac{t^2}{2}}.
\end{eqnarray}

The empirical cumulated density function \cite{artcomputerKnuthVol2} of a given
set of data $\{x_i\}_{i=1,N}$ is given by 
\begin{eqnarray}
F_N(x) &=& \frac{\textrm{number of } x_1,x_2,\ldots,x_n \textrm{ that are }\leq x}{N}.
\end{eqnarray}


The numerical method used by Scilab to generate such numbers is the Polar 
method for normal deviates, as presented in \cite{artcomputerKnuthVol2}.

\lstset{language=Scilab}
\lstset{numbers=left}
\lstset{basicstyle=\footnotesize}
\lstset{keywordstyle=\color{green}\bfseries}
\begin{lstlisting}
N=200;
x = rand(1,N,"normal");
Xsorted =gsort(x,"g","i"); 
Ydata = (1:N)/N;
plot(Xsorted,Ydata);
e=gce();
e.children.polyline_style=2;
xtitle("Empirical Cumulated Density Function of Normal Law with 200 samples")
filename = "introduction_ecdfnormal.png";
xs2png(0,filename);
\end{lstlisting}

The empirical cumulated density function 
is presented in figure \ref{introduction_ecdfnormal}.

\begin{figure}[htbp]
\begin{center}
\includegraphics[height=10cm]{introduction_ecdfnormal.png}
\end{center}
\caption{Empirical Cumulated Density Function of Normal Law with 200 samples}
\label{introduction_ecdfnormal}
\end{figure}

To compare the data which is produced by rand with the 
cumulated density function of the normal law, we use the 
\emph{cdfnor} primitive. This primitive is based on \cite{Algorithm715}
and uses rational functions that theoretically approximate the normal 
distribution function to at least 18 significant decimal digits. The same 
primitive can be used to compute the inverse of the cumulated density 
function. In that case, rational functions are used as starting values to 
Newton's Iterations which compute the inverse standard normal.

\lstset{language=Scilab}
\lstset{numbers=left}
\lstset{basicstyle=\footnotesize}
\lstset{keywordstyle=\color{green}\bfseries}
\begin{lstlisting}
N=200;
x = rand(1,N,"normal");
Xsorted =gsort(x,"g","i"); 
Ydata = (1:N)/N;
x=linspace(-3,3,100);
P=cdfnor("PQ",x,zeros(x),ones(x));
plot(Xsorted,Ydata,x,P);
\end{lstlisting}

The comparison plot between the empirical cdf and the 
computed cdf is presented in figure \ref{introduction_ecdcomparison}.

\begin{figure}[htbp]
\begin{center}
\includegraphics[height=10cm]{introduction_ecdfcomparison.png}
\end{center}
\caption{Cumulated Density Function of Normal Law : comparison of cdf from 
rational functions and empirical cdf from Polar method }
\label{introduction_ecdcomparison}
\end{figure}

The moments of a distribution can be computed with the 
\emph{mean}, \emph{variance} and \emph{stdev} Scilab macros,
which are implementations of the moments. For the variance
and standard deviation, the scaling factor is $N-1$.
In the following script, one computes these moment for 
an increasing number of samples, from $10^1$ to $10^5$.

\lstset{language=Scilab}
\lstset{numbers=left}
\lstset{basicstyle=\footnotesize}
\lstset{keywordstyle=\color{green}\bfseries}
\begin{lstlisting}
nbpoints = 5;
means=zeros(nbpoints,1);
vars=zeros(nbpoints,1);
stdevs=zeros(nbpoints,1);
nlist = 1:nbpoints;
for i = nlist
  N=10^i;
  x = rand(1,N,"normal");
  means(i) = mean(x);
  vars(i) = variance(x);
  stdevs(i) = stdev(x);
end
plot(nlist,[means,vars,stdevs]);
\end{lstlisting}

The convergence plot of the moments is presented in 
figure \ref{introduction_convergencemoments}.

\begin{figure}[htbp]
\begin{center}
\includegraphics[height=10cm]{introduction_convergencemoments.png}
\end{center}
\caption{Convergence of the moments of the normal law}
\label{introduction_convergencemoments}
\end{figure}



\chapter{Scilab statistical features}

In this chapter, we describe the features which are provided 
in Scilab's core, that is, "out of the box".
Indeed, Scilab provide features such as general statistical 
description of datas, many cumulated density functions 
and can generate uniform and non uniform random variates.
These features are based on several open source libraries, that 
we are analysing in the first section.
A complete overview of these features is provided in the 
second section, where we analyse the full list of functions and 
the numerical methods they use. For the most important 
functions, we provide a sample session where the function is 
used and some plots of the results.

\section{The sources}

In this section, we analyse the libraries which are available 
in Scilab and which provide its statistical features.
The figure \ref{inscilab-libraries} is an overview of the 
libraries which are either Scilab macros or source code, provided
in C, Fortran 77 or as Scilab macros. 

\begin{figure}[htbp]
\begin{tabular}{|l|l|}
\hline
Commands & calerf, erf, erfc, erfcx \\
Routines & CALERF\\
Directory & scilab/modules/elementary\_functions/src/fortran\\
Language & Fortran\\
Download & \url{http://www.kurims.kyoto-u.ac.jp/~ooura/index.html} \\
Author & Takuya Ooura \\
Year & 1996 \\
References & \cite{Algorithm715} \\
\hline
\hline
Name & Labostat \\
Directory & scilab/modules/elementary\_functions/src/fortran\\
Commands & General description functions (center, variance, etc...) \\
Language & Scilab scripts \\
Author & Carlos Klimann \\
Year & 2000 \\
References & \cite{Wonacott1990}, \cite{Saporta2006}\\
\hline
\hline
Name & DCDFLIB \\
Directory & scilab/modules/statistics/src/dcdflib\\
Download & \url{http://www.netlib.org/random/}\\
Commands & Cumulated Density Functions (cdfbet, cdfbin, etc...)  \\
Language & Fortran \\
Author & Barry Brown, W. J. Cody, Alfred H. Morris Jr \\
Year & 1994 for library, 1992 for code by Cody, 1991 for code by Morris \\
References & \cite{abramowitz+stegun1964}, \cite{HartEtAl:1968}, \cite{Algorithm715}, \cite{Kennedy1980}
\cite{Algo708}, \cite{DiDonato1986}\\
\hline
\hline
Name & Randlib \\
Directory & scilab/modules/randlib/src/fortran\\
Download & \url{ftp://odin.mda.uth.tmc.edu/pub/source}\\
& (unavailable at the time of the writing of this report)\\
Commands & grand (for distributions like normal, gamma, chi, etc...)  \\
Language & Fortran \\
Author & Barry Brown, James Lovato, Kathy Russell, John Venier \\
Year & 1997 \\
References & \cite{Ahrens1972}, \cite{358390}, \cite{Devroye86non-uniformrandom},
\cite{AhrensDieter1973}\\
\hline
\end{tabular}
\caption{Statistical libraries available in Scilab}
\label{inscilab-libraries}
\end{figure}


\section{Overview of functions}

The figure \ref{inscilab-fulllist} presents a complete list 
of Scilab statistical functions.

\begin{figure}[htbp]
\begin{tabular}{|l|l|}
\hline
\textbf{Description} & \\
\textbf{of Data} & \\
\hline
center &     cmoment \\
correl &     covar  \\
ftest &     ftuneq  \\
geomean &    harmean  \\
iqr &    labostat  \\
mad &    mean  \\
meanf &    median  \\
moment &    msd  \\
mvvacov &    nancumsum  \\
nand2mean &    nanmax  \\
nanmean &    nanmeanf  \\
nanmedian &    nanmin  \\
nanstdev &    nansum  \\
nfreq &    pca  \\
perctl &    princomp  \\
quart &    regress  \\
sample &    samplef  \\
samwr &    show\_pca  \\
st\_deviation &    stdevf  \\
strange &    tabul  \\
thrownan &    trimmean  \\
variance &    variancef  \\
wcenter & \\
\hline
\end{tabular}
\begin{tabular}{|l|l|}
\hline
\textbf{Special} & \\
\textbf{Functions} & \\
\hline
beta & calerf \\
erf & erfc \\
erfcx & erfinv \\
gamma & gammaln \\
\hline
\hline
\textbf{Random} &\\
\textbf{Number} &\\
\textbf{Generation} &\\
\hline
grand & prbs\_a \\
rand & sprand \\
randpencil &\\
\hline
\hline
\textbf{Cumulated} &\\
\textbf{Density} &\\
\textbf{Functions} &\\
\hline
cdfbet & cdfbin  \\
cdfchi & cdfchn  \\
cdff & cdffnc  \\
cdfgam & cdfnbn  \\
cdfnor & cdfpoi  \\
cdft & \\
\hline
\end{tabular}
\caption{Complete list of statistical features in Scilab}
\label{inscilab-fulllist}
\end{figure}

\subsection{General description functions}

The figure \ref{inscilab-descriptionfunctions} presents the 
general description functions available in Scilab.

\begin{figure}[htbp]
\begin{tabular}{|l|l|}
\hline
Name & Feature\\
\hline
center & center \\
wcenter & center and weight \\
cmoment & central moments of all orders \\
correl & correlation of two variables \\
covar & covariance of two variables \\
ftest & Fischer ratio \\
ftuneq & Fischer ratio for samples of unequal size. \\
geomean & geometric mean \\
harmean & harmonic mean \\
iqr & interquartile range \\
mad & mean absolute deviation \\
mean & mean (row mean, column mean) of vector/matrix entries \\
meanf & weighted mean of a vector or a matrix \\
median & median (row median, column median,...) of vector/matrix/array entries \\
moment & non central moments of all orders \\
msd & mean squared deviation \\
mvvacov & computes variance-covariance matrix \\
nancumsum & cumulative sum of the values of a matrix \\
nand2mean & difference of the means of two independent samples \\
nanmax & max (ignoring Nan's) \\
nanmean & mean (ignoring Nan's) \\
nanmeanf & mean (ignoring Nan's) with a given frequency. \\
nanmedian & median of the values of a numerical vector or matrix \\
nanmin & min (ignoring Nan's) \\
nanstdev & standard deviation (ignoring the Nans). \\
nansum & sum of values ignoring Nan's \\
nfreq & frequence of the values in a vector or matrix \\
pca & computes principal components analysis with standardized variables \\
perctl & computation of percentils \\
princomp & Principal components analysis \\
quart & computation of quartiles \\
regress & regression coefficients of two variables \\
sample & sampling with replacement \\
samplef & sample with replacement from a population and frequences of his values. \\
samwr & sampling without replacement \\
show\_pca & visualization of principal components analysis results \\
st\_deviation & standard deviation (row or column-wise) of vector/matrix entries  \\
stdevf & standard deviation \\
strange & range \\
tabul & frequency of values of a matrix or vector \\
thrownan & eliminates nan values \\
trimmean & trimmed mean of a vector or a matrix \\
variance & variance of the values of a vector or matrix \\
variancef & standard deviation of the values of a vector or matrix \\
\hline
\end{tabular}
\caption{Description of Data functions}
\label{inscilab-descriptionfunctions}
\end{figure}

\subsection{Special functions}

The figure \ref{inscilab-specialfunctions} presents the special functions 
available in Scilab.

\begin{figure}[htbp]
\begin{tabular}{|l|l|}
\hline
Name & Feature\\
\hline
beta & beta function \\
calerf & computes error functions \\
erf & error function \\
erfc & complementary error function \\
erfcx & scaled complementary error function \\
erfinv & inverse of the error function \\
gamma & gamma function \\
gammaln & logarithm of gamma function \\
\hline
\end{tabular}
\caption{Special functions}
\label{inscilab-specialfunctions}
\end{figure}

The figure \ref{inscilab-specialfunctionsdetailed} presents 
a detailed analysis of the location and internal design
of the special functions available in Scilab.

\begin{figure}[htbp]
\begin{tabular}{|l|l|}
\hline
Name & Location / Internals \\
\hline
beta & modules/special\_functions/sci\_gateway/c/sci\_beta.c \\
& switch to dgammacody by W. J. Cody and \\
& L. Stoltz and to betaln from DCDFLIB \\
\hline
calerf & modules/elementary\_functions/src/fortran \\
& by Takuya OOURA \\
\hline
erf & modules/elementary\_functions/macros/erf.sci \\
&  call to calerf \\
\hline
erfc & modules/elementary\_functions/macros/erfc.sci \\
& call to calerf \\
\hline
erfcx & modules/elementary\_functions/macros/erfcx.sci \\
& call to calerf \\
\hline
erfinv & modules/special\_functions/macros/erfinv.sci \\
& rational aproximation of erfinv + 2 Newton's steps\\
\hline
gamma & modules/special\_functions/sci\_gateway/fortran/sci\_f\_gamma.f  \\
& based on dgammacody by W. J. Cody and L. Stoltz \\
\hline
gammaln & modules/elementary\_functions/src/fortran/dlgama.f  \\
& by W. J. Cody and L. Stoltz \\
\hline
\end{tabular}
\caption{Detailed analysis of special functions}
\label{inscilab-specialfunctionsdetailed}
\end{figure}

\subsection{Cumulated density functions}

The figure \ref{inscilab-cdffunctions} presents the cumulated
density functions available in Scilab.

\begin{figure}[htbp]
\begin{tabular}{|l|l|}
\hline
Name & Feature\\
\hline
cdfbet & Beta distribution \\
cdfbin & Binomial distribution \\
cdfchi & chi-square distribution \\
cdfchn & non-central chi-square distribution \\
cdff & F distribution \\
cdffnc & non-central F distribution \\
cdfgam & gamma distribution \\
cdfnbn & negative binomial distribution \\
cdfnor & normal distribution \\
cdfpoi & poisson distribution \\
cdft & Student's T distribution\\
\hline
\end{tabular}
\caption{Cumulated density functions}
\label{inscilab-cdffunctions}
\end{figure}


\subsection{Random number generation}

The figure \ref{inscilab-randomnumbercommands} presents the random
number generators available in Scilab.

\begin{figure}[htbp]
\begin{tabular}{|l|l|}
\hline
Name & Feature\\
\hline
grand & Random number generators \\
prbs\_a & pseudo random binary sequences generation \\
rand & random number generator \\
sprand & sparse random matrix \\
randpencil & random pencil \\
\hline
\end{tabular}
\caption{Random number commands}
\label{inscilab-randomnumbercommands}
\end{figure}

The figure \ref{inscilab-randomnumbercommands} presents a detailed 
analysis of the location and design of the random
number generators available in Scilab.

\begin{figure}[htbp]
\begin{tabular}{|l|l|}
\hline
Name & Location / Internals \\
\hline
grand & modules/randlib/sci\_gateway/c/sci\_grand.c \\
& based on several random number generators\\
prbs\_a & modules/cacsd/macros/prbs\_a.sci \\
& based on rand \\
rand & modules/elementary\_functions/src/fortran/urand.f \\
& by Michael A. Malcolm And Cleve B. Moler \\
sprand & (todo)\\
randpencil & (todo)\\
\hline
\end{tabular}
\caption{Detailed analysis of random number commands}
\label{inscilab-detailrandomnumbercommands}
\end{figure}


\chapter{Statistical Toolboxes}

\url{http://www.scilab.org/contrib/index_contrib.php?page=download&category=DATA%20ANALYSIS%20AND%20STATISTICS}

GLMBOX :generalized statistical linear models analysis. (Dec 2003).
\url{http://www.scilab.org/contrib/index_contrib.php?page=displayContribution&fileID=183}

grocer 1.2 : Comprehensive econometric toolbox
\url{http://www.scilab.org/contrib/index_contrib.php?page=displayContribution&fileID=248}

Hurst : Exponent estimators v2.0
\url{http://www.scilab.org/contrib/index_contrib.php?page=displayContribution&fileID=988}

multilinear regression
\url{http://www.scilab.org/contrib/index_contrib.php?page=displayContribution&fileID=339}

Sci\_R for scilab 5.x
\url{http://www.scilab.org/contrib/index_contrib.php?page=displayContribution&fileID=1138}

stixbox 1.2.5
\url{http://www.scilab.org/contrib/index_contrib.php?page=displayContribution&fileID=184}
statistics toolbox designed for the french examination "agregation de mathematiques"


\chapter{Missing features}

\begin{itemize}
\item Empirical Cumulated Density Function
\item Robust implementation of variance, standard deviation.
See in "Art of Computer Programming" \cite{artcomputerKnuthVol2}, chapter 4.2.2, "Accuracy of Floating
Point Arithmetic", section A or in "Numerical Recipes" \cite{NumericalRecipes}, 
chapter 14.1, "Moments of a Distribution: Mean, Variance, Skewness, and so Forth".
\end{itemize}



\clearpage

%% Bibliography


\addcontentsline{toc}{chapter}{Bibliography}
\bibliographystyle{plain}
\bibliography{statisticsscilab}

\end{document}

