\documentstyle[11pt,verbatim,../tr2latex/troffman]{article}
             \textheight=660pt 
             \textwidth=15cm
             \topmargin=-27pt 
             \oddsidemargin=0.7cm
             \evensidemargin=0.7cm
             \marginparwidth=60pt
             \title{Tr2Tex for Scilab} 
             \author{J.Ph. Chancelier\thanks{Cergrene. Ecole Nationale des Ponts et Chauss\'ees, La Courtine  93167 Noisy le Grand C\'{e}dex }}
	
\begin{document}\maketitle
\def\verbatok#1{\expandafter\begin{verbatim}
\input{#1}}
	
\section{Modifications from the original tr2tex}
I've slightly modified the tr2tex program as follows~:
\begin{itemize}
\item The translation for \verb+.TH+ and \verb+.SH+ have been modified.
\verb+.SH NAME+ has a special translation, the line following this
statement must be of type 
\begin{verbatim}
<one-word> - <sentence>
\end{verbatim}
\item \verb+.TP+ and \verb+.IP+ are translated so as to obtain
description or itemize list in \LaTeX. The tag for \verb+.TP+ 
is by default in verb mode (tt font). One can't mix \verb+.TP+ and
\verb+.IP+ in the same enumeration ( the resulting code would not be 
properly indented )
\item \verb+.RS+ and \verb+.RE+ are used to produce indented sequences
of description or itemize~:
\begin{verbatim}
	.TP
	x
	1234567890
	.TP
	x
	1234567890
	.RS
	.IP x
	1234567890
	.IP y
	1234567890
	.RE
	.TP
	z
	1234567890
\end{verbatim}
\item troff Comments beginning with \verb+\"LaTeX+ are recognized and the
rest of the line is transmitted verbatim to \LaTeX. This allows the 
 writer to provide his own translations for the \LaTeX version of
the documentation and to add whatever statement he wants to the LaTeX 
documentation. \verb+.LA+ behaves like \verb+\"LaTeX+~:
\begin{verbatim}
	\"LaTeX \[ x=\sum_{k=0}^n \alpha_k^2 \]
	.LA \[ x=\sum_{k=0}^n \alpha_k^2 \]
\end{verbatim}
\item lines between \verb+.nf+ and \verb+.ni+ are supposed to be verbatim
statements and are translated as such. If the region surrounded with 
\verb+.nf+ and \verb+.ni+ is too large tr2tex can fail,
just cut your region into pieces.
\item \verb+.HR.+ is ignored in nroff and will produce an horizontal
line in \LaTeX
\item \verb+\fV+ will switch to \verb+tt+ font inside text \verb+.Vb+
will put is argument in \verb+tt+ font.
\end{itemize}
\section{An example : \LaTeX translation}
\input{example}
\section{An example : Troff code}
\verbatok{example.1}
\end{verbatim}


\end{document}
